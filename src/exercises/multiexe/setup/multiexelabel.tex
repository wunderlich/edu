\begin{macro}{\multiexelabel}
\begin{macro}{\multiexelabelboxed}
Diese Makros liefern die aktuelle Nummerierung der Teilaufgabe in Kleinbuchstaben mit anschließender Klammer (a), b),\dots) und erhöhen den Counter um Eins. \cs{multiexelabelboxed} setzt die Buchstaben zusätzlich rechtsbündig in eine Box.
\begin{MacroCode}{class}
\DeclareDocumentCommand \multiexelabel { s } {%
  \stepcounter{multiexecounter}%
  {\g_edu_multiexenumberstyle_tl%
    \textcolor{\g_edu_multiexenumberfg_tl}{%
      \IfBooleanT{#1} {
        \raisebox{0.1ex} {
          {\footnotesize$\star$}\hspace{0.75pt}
        }
      }  
      \g_edu_multiexenumberleft_tl\alph{multiexecounter}\g_edu_multiexenumberright_tl%
    }%
  }
}

\DeclareDocumentCommand \multiexelabelboxed { s } {%
  \stepcounter{multiexecounter}%    War \refstepcounter. Dies erzeugt vertikalen Abstand.
  \setlength{\fboxsep}{0pt}%
  \makebox[\g__edu_listlabelwidth_dim][r]{%
    \g_edu_multiexenumberstyle_tl%
    \textcolor{\g_edu_multiexenumberfg_tl}{%
      \IfBooleanT{#1} {
        \raisebox{0.1ex} {
          {\footnotesize$\star$}\hspace{0.75pt}
        }
      }   
      \g_edu_multiexenumberleft_tl\alph{multiexecounter}   
      \g_edu_multiexenumberright_tl
    }%
  }%
}

\end{MacroCode}
\end{macro}
\end{macro}

